\chapter{\IfLanguageName{dutch}{Stand van zaken}{State of the art}}
\label{ch:stand-van-zaken}

% Tip: Begin elk hoofdstuk met een paragraaf inleiding die beschrijft hoe
% dit hoofdstuk past binnen het geheel van de bachelorproef. Geef in het
% bijzonder aan wat de link is met het vorige en volgende hoofdstuk.

% Pas na deze inleidende paragraaf komt de eerste sectiehoofding.

Dit onderzoek zal zich focussen op z/OS Health Checker. Dit is een tool die draait in een mainframe omgeving. Vooraleer we kunnen beginnen met het bespreken van de oplossingsmethode van de onderzoeksvragen moeten we ons eerst verdiepen in de mainframe zelf en de omgeving waarin z/OS Health Checker draait om zo duidelijk te maken waarom deze tool zijn aanwezigheid belangrijk is en hoe deze werkt. Verder moeten we ook de basis begrijpen van andere tools en systemen binnen de mainframe zoals ISPF, SDS en JES. Om daarna te eindigen met JCL de taal die word gebruikt om de z/OS Health Checker logs op te stellen.

\section{De Mainframe}
\label{sec:De Mainframe}

De Mainframe speelt een centrale rol in de dagelijkse operaties bij de meeste grote bedrijven. De Ontwikkeling van de mainframe gaat terug tot de jaren '50. Ook al is er door de jaren heen veel veranderd aan de mainframe blijft het het meest stabiele,veilige en compatibele computing platform. Desondanks dat de mainframe een grote aanwezigheid heeft binnen de financiële wereld, blijft deze vrij onzichtbaar voor de grote menigte. Maar eigenlijk zijn we bijna allemaal indirect mainframe gebruikers ook al realiseren we het niet. (\cite{Ebbers2011})

De term mainframe kan vandaag het beste beschreven worden als een stijl van operaties, applicaties en besturing systeem faciliteiten. Een definitie hiervan zou zijn: "Een mainframe is hetgeen dat bedrijven gebruiken voor het hosten van commerciële databanken, transactie servers en applicaties die een hogere graad van security en availability nodig hebben dan die in machines van een kleinere schaal. 
De term mainframe is duidelijk vervaagd met de jaren daarom wordt de term meestal geassocieerd met een systeem met volgende eigenschappen. (\cite{Ebbers2011})

\begin{itemize}
	\item Compatibiliteit met System z besturingssystemen, applicaties en data.
	\item Gecentraliseerde controle van resources
	\item Hardware en besturingssystemen die toegang kunnen delen tot disk drives met ander systemen met automatische locking en bescherming tegen destructief simultaan gebruik van data.
	\item Hardware en besturingssystemen die continu werken met honderden of duizenden simultane input-output operaties
	\item Clustering technologieën bevatten zoals de Parallel Sysplex die de mainframe flexibel en schaalbaar maken
	\item Geoptimaliseerd voor input-output business gerelateerde data processing applicaties
\end{itemize}

\begin{figure}[h]
	\centering
	\includegraphics{img/mainframe}
	\caption[Mainframe z13 en LinuxOne Rockhopper]{Een paar mainframes, links de IBM z systems z13 en rechts de LinuxOne Rockhopper}
	\label{fig:mainframe}
\end{figure}



\section{Logische partities en de Parallel Sysplex}
\label{sec:Logische partities en de Parallel Sysplex}

Nu men weet wat een mainframe is, is het belangrijk om te kijken naar de architectuur binnen het systeem zelf. Dit is aan de hand van allerlei gekoppelde logische partities die men als geheel de Parallel Sysplex noemt. Dit is ook de omgeving waarin z/OS Health Checker opereert. Daarom belichten we ook deze componenten in dit hoofdstuk.

\subsection{Logische partitie of LPAR}
\label{subsec:Logische partitie of LPAR}
De IBM mainframe kan verdeeld worden in verscheidene logische systemen. Tussen deze systemen kan men volgende resources verdelen:
\begin{itemize}
	\item Memory
	\item Processors
	\item Input-Output devices.
\end{itemize}
Deze aparte systemen noemt men een logische partitie of LPAR. Al deze LPARs staan onder de controle van een hypervisor. De hypervisor is een software laag voor het beheer van meerdere besturingssystemen. De Verdeling van resources gebeurt door de Processor Resource/Systems Manager(PR/SM). De volledige definitie van een LPAR luidt als volgt: Een subset van de processor hardware dat gedefinieerd is voor het ondersteunen van een besturingssysteem. Meerdere LPARs zijn dus gelijkaardig aan verschillende aparte mainframes. Ze hebben elk hun eigen besturingssysteem en toegewezen hardware. (\cite{Ebbers2011})

\begin{figure}[h]
	\centering
	\includegraphics{img/LPAR}
	\caption[Logische Partities]{Visualisatie van logische partities}
	\label{fig:lpar}
\end{figure}


\subsection{Parallel Sysplex}
\label{subsec:Parallel Sysplex}

De Parallel Sysplex is een een techniek van clusteren. Waarmee men meerdere LPARs groepeert.  z/OS Health Checker zal zowel opereren op aparte LPAR als op de gehele Parallel Sysplex door globale checks(meer hierover in sectie \ref{sec:z/OS Health Checker}). Daarvoor gaan we ons ook verdiepen in deze clustering techniek.

Sysplex staat voor SYStems comPLEX dit is een of meerdere LPARs met z/OS, samengevoegd als 1 unit die gespecialiseerde hardware en software gebruikt. Het gebruikt unieke messaging services en kan bestandsstructuren delen in de couple facilty(CF) datasets. Een sysplex is een instantie van een computer systeem dat draait op 1 of meerdere fysieke partities waarvan elke een andere release kan draaien van het z/OS besturingssysteem. Een sysplex is wel geïsoleerd tot 1 fysieke mainframe. De Parallel Sysplex anderzijds laat meerdere mainframes zich voordoen als 1 systeem. (\cite{Ebbers2011}) 

Een Parallel Sysplex is een symmetrische sysplex die gebruik maakt van het delen van data met meerdere systemen. Dit is dus de clustering van meerder mainframes. We bespreken ook enkele protocollen die de Parallel Sysplex gebruikt. 

\subsubsection{Server Time Protocol}
\label{subsubsec:Server Time Protocol}

Een belangrijk aspect van de Parallel Sysplex is het synchroniseren van de Time Of Day(TOD) klokken van de meerdere servers. Stel nu meerdere systemen hebben net in dezelfde database data aangepast, maar daarna gebeurt er een uitval. Dan zal men de databank reconstrueren met behulp van alle timestamps van alle aanpassingen. Hiervoor is het belangrijk dat de klokken van elke LPAR gesynchroniseerd zijn om zo de juiste data in de juiste volgorde te reconstrueren. Dit gebeurt vandaag met het Server Time Protocol(STP). (\cite{Ebbers2011})

\subsubsection{Coupling Facilty}
\label{subsubsec:Coupling Facility}

Sommige z/OS applicaties op verschillende LPARs hebben vaak toegang nodig to dezelfde informatie. Hiervoor betrouwt een Parallel Sysplex op een of meerder Coupling Facilities(CF). Een CF maakt het mogelijk om aan data sharing te doen met meerdere systemen. Een CF is ook een LPAR maar een speciale die andere LPARs toelaat data te delen. (\cite{Ebbers2011})

\begin{figure}[h]
	\centering
	\includegraphics{img/ParallelSysplex}
	\caption[Visualisatie van een Parallel Sysplex]{Visualisatie van een parallel sysplex met 2 LPARs en 1 Coupling Facility. Control Units controleren de logica voor bepaalde I/O-apparaten zoals printers of opslagfaciliteiten.}
	\label{fig:parallelsysplex}
\end{figure}

Een goed geconfigureerde parallel sysplex cluster is zodanig ontworpen dat het een hoge beschikbaarheid biedt met een minimale onbeschikbaarheid. Dit is dus al een van technieken die de mainframe zijn hoge availabilty garanderen. Bijvoorbeeld: wanneer een system uitvalt, dan kan een ander systeem dit direct opvangen omdat alle data en kritische applicaties gedeeld worden in de parallel sysplex. De taken van het uitvallende systeem worden overgenomen. Ondertussen kan het gefaalde systeem herstart worden. Dit zorgt er uiteindelijk voor dat een laag aantal single points of failure aanwezig is binnen de sysplex. (\cite{Ebbers2011})

\section{z/OS}
\label{subsec:z/OS}

Een ander belangrijk component van elk systeem is het besturingssysteem. In het geval van deze proef is dat z/OS. Dit niet het enige maar wel het meest gebruikte besturingssysteem op de mainframe. Zoals de naam het zelf zegt draait ook z/OS Health Checker op dit besturingssysteem. Daarom bespreken we ook dit besturingssysteem. Een besturingssysteem is eigenlijk een collectie van programma's die de interne werking van het computer systeem beheren. Een besturingssysteem is ontworpen om er voor te zorgen dat de resources van de computer optimaal gebruikt worden.

z/OS is vandaag een resultaat van tientallen jaren technologische vooruitgang door IBM. Het begon als een besturingssysteem dat maar 1 programma tegelijk kon afhandelen naar een dat vandaag duizenden programma's en gebruikers tegelijk kan afhandelen. Het besturingssysteem wordt uitgevoerd in de processor en bevindt zich ook in de processor storage. Mainframe hardware bestaat uit een aantal processors en gekoppelde toestellen zoals DASD(Direct Acces storage devices de mainframe term voor hard disks). Die worden dan allemaal aangestuurd vanuit de consoles gekoppeld aan de mainframe. De DASD's worden gebruikt voor systeem functies of door programma's van gebruikers die uitgevoerd worden door z/OS. (\cite{Ebbers2011})

\begin{figure}[h]
	\centering
	\includegraphics{img/Omgeving_zOS}
	\caption[z/OS binnen de mainframe omgeving]{z/OS binnen de hardware omgeving bevindt zich op de mainframe. En wordt aangestuurd door aangesloten consoles/terminals. Om dan de data te bewerken op tape drives of DASD(term voor hard disk binnen de mainframe wereld)}
	\label{fig:omgevingzos}
\end{figure}

z/OS maakt het ook mogelijk om aan multiprocessing en multiprogramming te doen. Hierdoor maakt het z/OS geschikt voor het uitvoeren van programma's die veel input/output operaties nodig hebben. 

\subsubsection{Multiprocessing}
\label{subsubsec:Multiprocessing}
Dit is het simultaan opereren van meerdere processors die meerdere hardware resources delen zoals memory of externe opslag

\subsubsection{Multiprogramming}
\label{subsubsec:Multiprogramming}
Multiprogramming laat z/OS toe om duizenden programma's te draaien voor gebruikers die werken aan verschillende projecten waar men zich ook bevindt op de wereld.

\subsection{Storage gebruik van z/OS}
\label{subsec:Storage gebruik van z/OS}

z/OS heeft ook zijn eigen manier voor het gebruik van opslag/storage. Dit is ook belangrijk, omdat de term 'address space' gebruikt zal worden binnen deze proef. En die term is de techniek die z/OS gebruikt binnen de storage omgeving.

Een mainframe en een gewone computer hebben 2 soorten fysieke opslag:

\begin{itemize}
	\item Fysieke storage die zich bevindt op de mainframe processor zelf, ook wel 'real storage' genoemd te vergelijken met RAM op je laptop.
	\item Fysieke storage die zich buiten de mainframe bevindt op een tape of disk drive, dit wordt de auxiliary storage genoemd.
\end{itemize}

z/OS gebruikt deze 2 soorten storage om een andere soort te vormen namelijk virtual storage. Virtual storage is een combinatie van real en auxiliary storage. Het gebruikt een serie van tabellen en indexen om locaties binnen het real geheugen te associëren met locaties in de auxiliary storage.

Een address space is eigenlijk een range van virtuele adressen dat het besturingssysteem toekent aan een programma of gebruiker. Voor een gebruiker kan dit beschouwd worden als een container waar zijn data in zit. Door deze addres space moet z/OS niet een heel programma naar de real storage laden om het uit te voeren. Daarom zal men het programma in stukken(ook gekend als pages) van de auxiliary storage naar de real storage verplaatsen in de volgorde die nodig is om het prorgamma uit te voeren. Eens dat een page niet meer nodig is kan men het terugschrijven naar de auxiliary storage. Dit laat z/OS toe om meer programma's simultaan uit te voeren.

De fysieke opslag is daarom opgedeeld in verschillende stukken die elke hun eigen adres hebben maar de pages worden opgevraagd met hun virtueel adres. Het proces om een virtueel adres te vertalen in een real address noemt men Dynamic Address translation(DAT). (\cite{Ebbers2011})

\begin{figure}[h]
	\centering
	\includegraphics{img/Storage}
	\caption[Visualisatie van het concept van virtuele storage]{Bijna alle programma's gebruiken virtuele adressen als ze refereren naar data in de real storage. Maar als een aangevraagd adres zich niet in de real storage bevindt zal er een onderbreking plaatsvinden en zal men de nodige data uit auxiliary storage naar de real storage laden(ook wel paging genoemd)}
	\label{fig:storage}
\end{figure}

Een andere manier om te denken aan een address space is eigenlijk een soort van kaart voor de programmeur waarmee hij al zijn code en data kan opvragen.

\section{Interactie met z/OS}
\label{sec:interactie met z/OS}

Om een systeem te beheren moeten we er natuurlijk ook mee kunnen werken. Voor deze interactie gebruikt men bijvoorbeeld een TSO/E commando om aan te loggen en dan ISPF, een collectie van menu's en panels die brede range van functies aanbied.

\subsection{Time Sharing Option/Extensions en Interactive System Productivity Facility}
\label{subsec:Time Sharing Option}

TSO/E of Time Sharing Option/Extensions laat gebruikers toe om een interactieve sessie te maken met een z/OS systeem. Hierdoor kunnen ze aanloggen op het z/OS systeem en gebruik maken van een command prompt interface. Maar omdat de command prompt niet echt handig is wordt er meestal gebruik gemaakt van de Interactive System Productivity Facility (ISPF). Dit is een collectie van menu's en panelen die een wijde range van functies aanbied voor het bewerken van data in z/OS. Zo bied ISPF onder andere een tekst editor aan en functies voor het vinden en oplijsten van bestanden. (\cite{Parziale2017})

\begin{figure}
	\centering
	\includegraphics[width=0.7\linewidth]{img/IPSF}
	\caption[ISPF hoofdmenu]{Dit is het hoofdmenu van ISPF vanaf hier kan je de verschillende panelen gebruiken}
	\label{fig:ispf}
\end{figure}


De bestanden binnen z/OS worden ook wel 'data sets' genoemd, dit is de term die ook verder gebruikt zal worden als we over bestanden spreken binnen z/OS. Er zijn 2 soorten data sets waarmee gewerkt word in deze proef.

\begin{itemize}
	\item Sequential data set: dit zijn data sets waarvan de individuele records georganiseerd zijn op hun fysieke volgorde binnen de dataset.
	\item Partitioned data set(PDS): dit is een data set die verdeeld is in partities ook wel 'members' deze kunnen een programma bevatten of gewoon data. Eigenlijks is een PDS een collectie van sequentiële data sets. Je zou het dus kunnen vergelijken met een folder met bestanden.
\end{itemize}

\section{Job Control Language en Job Entry Subsystem}
\label{sec:Job Control Language en Job Entry Subsystem}

In deze proef worden ook Jobs uitgevoerd voor het opstellen van de logs van z/OS Health Checker. Hiervoor wordt de Job Control Language(JCL) gebruikt. Want om een prorgamma uit te voeren moet het verwerkt worden door z/OS.

Vooraleer het z/OS systeem een programma voor je kan uitvoeren moet je een paar dingen doen. Je moet eerst natuurlijk beschrijven welk programma je wil uitvoeren maar ook welke resources er gebruikt zullen worden(bv: eventuele input). Dit doe je met een JCL job. Deze jobs 'submit' je dan naar het Job Entry Subsystem(JES) deze zal de jobs inplannen en hun output verwerken. (\cite{Cosimo2018})

\begin{figure}[h]
	\centering
	\includegraphics{img/JES}
	\caption[Visualisatie van JCL en JES]{De gebruiker definieert, maakt en zal een job submitten. Deze wordt dan verwerkt door het Job Entry Subsystem en zal de output van de job teruggeven als resultaat.}
	\label{fig:jes}
\end{figure}

\section{System Display and Search Facility}
\label{sec:System Display and Search Facility}

We willen natuurlijk de output van onze jobs kunnen zien daarvoor wordt er gebruikt gemaakt van de System Display and Search Facility of SDSF. Dit is niet het enige dat we kunnen zien in SDSF. Voor z/OS Health Checker zullen we ook deze tool gebruiken om alle info van het systeem op te vragen. Deze functie kan je bereiken via het hoofdmenu van ISPF met de optie 'S'(zie figuur \ref{fig:ispf} )

In SDSF zijn er veel panelen en soorten output die je kan opvragen maar de enige die wij zullen gebruiken is namelijk die van z/OS Health Checker die bereik je binnen SDSF via de 'ck' optie. Verder zal je bij het schrijven van jobs het 'Status of jobs' paneel nodig hebben. Dit panel toont de output van uitgevoerde jobs, dit paneel bereik je met de 'st' optie.

\begin{figure}[h]
	\centering
	\includegraphics[width=0.7\linewidth]{img/SDSF}
	\caption[SDSF Hoofdmenu]{Hier zie je het hoofdmenu van SDSF en onder andere ook de 2 opties die gebruikt worden doorheen de proef namelijk 'ck' en 'st'}
	\label{fig:sdsf}
\end{figure}

 
\section{De Parmlib}
\label{sec:De Parmlib}

z/OS Health Checker heeft ook enkele datasets in de parmlib die veranderd worden in deze proef daarom moeten we ook begrijpen wat die parmlib eigenlijk is. Elk z/OS systeem heeft een PDS met members die worden meegegeven door IBM. Deze PDS is SYS1.PARMLIB. De members zijn allemaal systeem en applicatie parameters die het systeem nodig heeft bij het opstarten(Initial Program Load(IPL) in mainframe term). Bij de IPL wordt de parmlib gelezen om het systeem op te zetten. Deze PDS wordt later ook nog gelezen door andere componenten en programma's. (\cite{Cosimo2018})
 
Een van deze componenten is z/OS Health Checker. In de parmlib zitten members die definiëren welke controles (Checks) er zullen uitgevoerd worden en welke niet.
 
\section{z/OS Health Checker}
\label{sec:z/OS Health Checker}

Na een analyse naar de oorzaken van de verschillende uitvallen kwam men tot de conclusie dat veel hiervan perfect vermeden had kunnen worden. Vele uitvallen kwamen door slechte configuraties die leiden tot single points of failure. Hierdoor is z/OS Health Checker ontwikkelt door IBM. (\cite{Walle2013})

IBM Health Checker voor z/OS is een tool die helpt om potentiële problemen op te sporen in de configuratie van het systeem. Deze problemen zouden een grote impact kunnen hebben op het systeem of zouden zelfs een uitval kunnen veroorzaken. Health Checker kijkt de huidige instellingen van z/OS en de Sysplex na en vergelijkt deze met instellingen die door IBM aangeraden worden. Bij eventuele problemen zal Health Checker een output genereren met gedetailleerde info over het probleem zelf en op welke manier je het probleem het beste oplost. Wel belangrijk is dat Health Checker eerder een preventieve tool is en geen monitoring tool. (\cite{Bezzi2010})

z/OS Health Checker bestaat uit 2 delen
\begin{itemize}
	\item De checks
	\item Het framework	
\end{itemize}

\subsection{Health Checker Framework}
\label{subsec:Health Checker Framework}

Het framework van z/OS Health Checker is de interface die je gebruikt om checks uit te voeren en te beheren. Deze ondersteunt niet enkele checks van IBM maar ook die van software producenten zoals Computer Associates en Compuware. Binnen het framework zit de HZSPQE data, hierin zit alle informatie die een check nodig zou hebben zoals de default parameters.  Het framework bevat ook de Message table die de data bijhoudt van de output van de checks. (\cite{IBMCorporation2019})

\begin{figure}[h]
	\centering
	\includegraphics[width=0.7\linewidth]{img/HCFramework}
	\caption[z/OS Health Checker Framework]{Het z/OS Health Checker Framework dat het mogelijk maakt om checks uit te voeren en de output hiervan door te geven aan SDSF en andere tools.}
	\label{fig:hcframework}
\end{figure}


\subsection{De Checks}
\label{sec:z/OS Health Checker Checks}

De Checks zelf zijn programma's die componenten en instellingen evalueren en dan eventuele problemen melden. Deze checks worden voornamelijk aangeboden door IBM zelf maar deze kunnen ook van andere bedrijven zijn. Je kan ook zelf Checks schrijven met System REXX. Een check vergelijkt de instellingen van 1 bepaald component met een set van instellingen die aangeraden worden door de eigenaar van de check. De aangeraden instellingen zijn geen verplichte instellingen maar eerder een best practice van IBM. Je kan deze ook zelf definiëren of aanpassen. (\cite{Bezzi2010})

Een check wordt gedefinieerd met 3 waarden

\subsubsection{Check Owner}
\label{subsubsec:Check Owner}

Elke check heeft zijn eigenaar. De naam van de check is meestal verbonden met het component waarvoor hij draait. In deze naam zit ook bijna altijd een verwijzing naar het bedrijf of product die de check gemaakt heeft. De checks van IBM beginnen dan ook allemaal met IBM. Een check owner is een string van maximum 16 karakters. Een voorbeeld van een check owner is bijvoorbeeld IBMXCF. Deze check is dus geschreven door IBM en zal iets te maken hebben met XCF(Cross-system Coupling Facility). Een check owner heeft meerdere checks zelf. (\cite{Bezzi2010})

\subsubsection{Check Name}
\label{subsubsec:Check Name}

De check zelf heeft natuurlijk ook een naam. En is uniek voor elke check. Deze zal maximum 32 karakters lang zijn. Een voorbeeld van een check naam: "XCF\_CDS\_MAXSYSTEM".

\subsubsection{Check Values}
\label{subsubsec:Check Values}

Een check bezit verder ook nog enkele vooraf gedefinieerde waarden

\begin{enumerate}
	\item Een interval die definieert hoe vaak en wanneer de check uitgevoerd wordt.
	\item De ernst(Severity) van een check deze definieert de check output. Hoe grotere het potentiele probleem hoe hoger de severity. 
\end{enumerate}

\subsubsection{Check Types}
\label{subsubsec:Check Type}

Verder zijn er ook 3 types van checks: Local, remote en REXX checks. In deze proef komen enkel de local checks aan bod. De lokale check is geschreven in ASSEMBLER. En draaien binnen de Health Checker address space. Deze worden opgeroepen met parameters die verwijzen naar de HZSPQE data hierin zit alle data die de check routine nodig heeft. Het verschil met een Remote check is dat een remote check niet binnen de addres space van Health Checker draait. 

\subsubsection{Check Output}
\label{subsubsec:Check Output}

De check output is ook zeer belangrijk deze duid namelijk aan of een check al dan niet succesvol was. Er zijn verschillende soorten check messages.

\begin{itemize}
	\item Information Message: Deze message krijg je wanneer een check succesvol is of wanneer deze niet kan draaien in de huidige omgeving(Bijvoorbeeld wanneer het component dat de check evalueert niet aanwezig is op het systeem).
	\item Exception Message: Deze message krijg je wanneer de check onsuccesvol was en deze een potentieel probleem heeft gevonden. Dit noemt een exception. Dit bericht bevat de severity van het probleem samen met suggesteis om het probleem op te lossen
	\item Report: Bij een exception zal er ook een extra report bijgevoegd zijn met extra informatie over het probleem.
	\item Debug: Sommige checks kan je in debug mode uitvoeren. Dit wordt vooral enkel gebruikt bij het ontwikkelen van een eigen check.	
\end{itemize}

Voor een voorbeeld van check output zie bijlage \ref{sec:Check Output}

\subsubsection{Check Status}
\label{subsubsec:Check Status}

Een check heeft ook een bepaalde status. Deze status duid aan of dat een check al dan niet geactiveerd is. Volgende statussen zijn mogelijk. (\cite{IBM2019})

\begin{itemize}
	\item ACTIVE: Dit specifieert dat de check actief is en draait.
	\item INACTIVE: Dit specifieert dat de check niet actief is en deze niet zal draaien.
	\item Ookal is een check actief of inactief kan deze ook nog 2 opties hebben namelijk Enabled \& Disabled. Wanneer een check enabled is betekent dat dat deze kan draaien binnen de huidige omgeving. Als een check disabled is kan dit betekenen dat deze niet kan draaien omdat er condities binnen de huidige parallel sysplex omgeving zijn die dat verhinderen. Bijvoorbeeld een check van DB2 die disabled staat omdat er om een bepaalde LPAR geen DB2 product is.
	\item GLOBAL: Dit betekent dat de check globaal is en daarom maar op 1 LPAR mag draaien van de gehele sysplex. Maar de check zal wel instellingen controleren voor de hele parallel sysplex. 
\end{itemize}


Verder is er nog 1 speciaal soort check. Namelijk de migration check. Deze check helpen je bij het plannen voor een overschakelen van z/OS versie. Deze checks kan je uitvoeren na een migratie om te kijken of deze succesvol was. Deze checks beginnen altijd met 'ZOSMIG'. (\cite{IBMCorporation2019})
