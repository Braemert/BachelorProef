%%=============================================================================
%% Inleiding
%%=============================================================================

\chapter{\IfLanguageName{dutch}{Inleiding}{Introduction}}
\label{ch:inleiding}

Men zal het niet beseffen maar zonder de Mainframe zouden veel hedendaagse diensten wegvallen. Ook al is de mainframe voor velen iets uit het verleden, veel sectoren blijven er op vertrouwen. Een voorbeeld zijn de bankinstellingen, de kans is groot dat als je iets betaalt met je bankkaart dat de transactie verwerkt wordt door een mainframe. Een ander voorbeeld zijn de vliegmaatschappijen, als je eens een kijkje zou kunnen nemen naar de computer die gebruikt wordt op de luchthaven om je in te checken op je vlucht, zal je merken dat dit hoogstwaarschijnlijk een terminal is die aangesloten zit op een Mainframe. Verder wordt de mainframe ook nog gebruikt in andere sectoren zoals: Financiële Sector, Magazijnbeheer, Verzekeringen, Ziekenzorg, Overheid, etc.

Een uitval van een mainframe kan dus kritische diensten laten wegvallen. Bijvoorbeeld de mogelijkheid tot overschrijven via de bank. Daarom is een van de belangrijkste factoren van de mainframe de beschikbaarheid. De mainframe garandeert een uptime van 24 op 24, 7 op 7, 365 op 365 het hele jaar door dus. Hiervoor worden verscheidene technieken gebruikt zoals de "alles 2 regel". Je zult in een mainframe alles dubbel terugvinden dit zorgt voor redundantie zo zal je in een mainframe kast 2 laptops vinden om met de mainframe te communiceren. Ook de architectuur binnen de mainframe, de Parallel Sysplex(Meer hierover in hoofdstuk \ref{ch:stand-van-zaken}) zorgt dat bij een uitval van 1 partitie een andere zijn workload direct overneemt. Dit zorgt voor hoge stabiliteit binnen de mainframe. Zo ligt hun mean time between failures\footnote{MTBF is een methode om de betrouwbaarheid van een systeem te meten} meting in de tientallen jaren. De mainframe garandeert ook een hoog niveau van security. Dan zijn er binnen die architectuur ook nog verschillende software componenten die hier voor zorgen. Een daarvan is degene waar deze proef zich op focust z/OS Health Checker, deze tool werkt preventief. Het zal proberen om de problemen op te sporen alvorens die plaats vinden en zal de System Administrator hiervan verwittigen. In hoofdstuk \ref{ch:stand-van-zaken} zal de werking hiervan tot in detail worden uitgelegd.

\section{\IfLanguageName{dutch}{Probleemstelling}{Problem Statement}}
\label{sec:probleemstelling}

De z/OS Health Checker opstelling van HCL Technologies is al lang niet meer veranderd en is ook niet gestandaardiseerd over de verschillende partities binnen de Mainframe. Bij eventuele uitbreiding van het systeem is er dus ook niet direct een standaardopstelling die men kan toepassen. Verder vind er ook een inefficiënte logging plaats. Deze is nu niet echt aanwezig, de bedoeling is dat de verantwoordelijke van elke partitie een log krijgt van alle fouten binnen zijn partitie.

\section{\IfLanguageName{dutch}{Onderzoeksvraag}{Research question}}
\label{sec:Hoofdonderzoeksvraag}

Deze proef zal zich focussen op de z/OS Health checker opstelling van HCL technologies met volgende onderzoeksvraag.

\begin{itemize}
	\item Kan een huidige Health Checker opstelling geoptimaliseerd worden zodat het beheer hiervan gecentraliseerd is \& gestandaardiseerd worden zodat deze bij een uitbreiding van het systeem, de opstelling toegepast kan worden?
\end{itemize}

\subsection{Deelonderzoeksvragen}
\label{subsec:Deelonderzoeksvragen}

Daarnaast is er in deze proef ook een focus op verdere efficiëntere logging van z/OS Health Checker en of deze kan via een webUI. Daaruit volgen de deelonderzoeksvragen:

\begin{itemize}
	\item Kan er een effectiever log-systeem opgezet worden voor de z/OS Health Checker output?
	\item Is het mogelijk een log systeem op te stellen via een WebUI?
\end{itemize}


\section{\IfLanguageName{dutch}{Onderzoeksdoelstelling}{Research objective}}
\label{sec:onderzoeksdoelstelling}

Het doel van deze proef is om een standaardopstelling te bekomen van z/OS Health Checker binnen de omgeving van HCL Technologies zodat deze bij uitbreiding van het systeem deze onmiddellijk kan implementeren op nieuwe partities.

Verder is het doel om ook een duidelijke logging op te stellen voor elke verantwoordelijke van elke logische partitie van de Mainframe omgeving van HCL, met als eventueel vervolg een mogelijkheid te vinden om die logging te laten gebeuren via een web interface.

\section{\IfLanguageName{dutch}{Opzet van deze bachelorproef}{Structure of this bachelor thesis}}
\label{sec:opzet-bachelorproef}

% Het is gebruikelijk aan het einde van de inleiding een overzicht te
% geven van de opbouw van de rest van de tekst. Deze sectie bevat al een aanzet
% die je kan aanvullen/aanpassen in functie van je eigen tekst.

De rest van deze bachelorproef is als volgt opgebouwd:

In Hoofdstuk~\ref{ch:stand-van-zaken} wordt een overzicht gegeven van de stand van zaken binnen het onderzoeksdomein, op basis van een literatuurstudie.

In Hoofdstuk~\ref{ch:methodologie} wordt de methodologie toegelicht en worden de gebruikte onderzoekstechnieken besproken om een antwoord te kunnen formuleren op de hoofdonderzoeksvraag en de 1ste deelonderzoeksvraag.

In hoofdstuk~\ref{ch:Vervolg op bachelorproef} word er gekeken naar eventuele vervolgen voor deze bachelor proef.


% TODO: Vul hier aan voor je eigen hoofstukken, één of twee zinnen per hoofdstuk

In Hoofdstuk~\ref{ch:conclusie}, tenslotte, wordt de conclusie gegeven en een antwoord geformuleerd op de onderzoeksvragen. Daarbij wordt ook een aanzet gegeven voor toekomstig onderzoek binnen dit domein.