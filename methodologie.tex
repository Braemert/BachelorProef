%%=============================================================================
%% Methodologie
%%=============================================================================

\chapter{\IfLanguageName{dutch}{Methodologie}{Methodology}}
\label{ch:methodologie}

%% TODO: Hoe ben je te werk gegaan? Verdeel je onderzoek in grote fasen, en
%% licht in elke fase toe welke stappen je gevolgd hebt. Verantwoord waarom je
%% op deze manier te werk gegaan bent. Je moet kunnen aantonen dat je de best
%% mogelijke manier toegepast hebt om een antwoord te vinden op de
%% onderzoeksvraag.



Na de stand van zaken die gebaseerd is op de literatuurstudie volgt de volgorde van stappen die ondernomen zijn om deze proef te voltooien. 

\section{z/OS Health Checker standaard setup}
\label{sec:z/OS Health Checker Standaard setup}

De eerste onderzoeksvraag was of er mogelijkheid was tot een standaardopstelling binnen de z/OS Health Checker omgeving van HCL Technologies. Maar eerst moet er een analyse plaatsvinden op de huidige opstelling van Health Checker. Om deze te optimaliseren

\subsection{Analyse van huidige z/OS Health Checker Setup}
\label{subsec:Analyse van huidige z/OS Health Checker Setup}
De opstelling die in deze proef geanalyseerd werd bevind zich binnen een parallel sysplex. En deze parallel sysplex werken we met 4 LPARS: VT1, VT2, VT3 en VT4. Elke LPAR heeft verschillende checks. Maar na de 4 LPARS te overlopen was het duidelijk dat de meeste checks op VT1 draaien. Daarom is de analyse gefocust op VT1.

De analyse is gemaakt met de check data uit SDSF deze kan je bereiken door bij het ISPF hoofdmenu volgende optie te geven 's;ck'. Dit is S voor SDSF met als volgende optie CK voor Health Checker deze ziet er zo uit.

\begin{figure}[h]
	\centering
	\includegraphics[width=0.7\linewidth]{img/SDSFCK}
	\caption[Health Checker Scherm binnen SDSF]{Dit is het Health Checker paneel binnen SDSF hier kunnen we de output van de laatste keren dat een check is uitgevoerd}
	\label{fig:sdsfck}
\end{figure}


Na het overlopen van alle checks op VT1 hebben we de SDSF ouptut samengevat in volgende tabel. Met de naam van de check. De status, deze beschrijft of de check aanstaat of niet. De outcome, deze beschrijft of de check succesvol was of niet. En de reason, dit is de reden waarvoor de check draait. Een voorbeeld:
\begin{table}
	\begin{tabular}{|p{5cm}|p{3.5cm}|p{1.5cm}|p{5cm}|}
		\hline
		\textbf{Name} & \textbf{Status} & \textbf{Outcome} & \textbf{Reason} \\
		\hline
		XCF\_CDS\_MAXSYSTEM & ACTIVE(ENABLED) & SUCCES & CDS MAXSYSTEM value across all CDS types should be at least equal to the value 
		in the primary sysplex CDS.  \\
		\hline
	\end{tabular}
	\caption[Individuele check]{Waarden van de check die bij de eerste analyse werden samengevat}
	\label{tbl:Individuele check}
\end{table}


Voor de volledige tabel zie bijlage 

Hier uit bleek dat er verscheidene check een exception hadden. Deze zijn eerst gecontroleerd of dat men deze zelf kon oplossen zonder ondersteuning van de collega's die het product beheren. Bij deze analyse bleek ook dat er veel checks zijn die onnodig draaiden omdat het product van de check bijvoorbeeld niet aanwezig was. Niet alle checks moeten in alle LPARs draaien en sommige moesten aangepast worden. Maar daarvoor heb je nog de belangrijke vraag of een wijziging van een check doorgevoerd moet worden naar alle LPARs of maar naar 1 LPAR binnen de Parallel Sysplex. Met deze informatie is naar feedback gevraagd van alle Mainframe Teams binnen HCL Technologies. Want men kan checks niet aanpassen van bijvoorbeeld DB2 zonder dit raad te plegen met het DB team. Hiervoor zijn alle checks gegroepeerd per owner daarna nog eens gegroepeerd per mainframe team en uiteindelijk werd tabel \ref{tbl:Checks Per Team} bereikt.

\begin{table}[]
	\begin{tabular}{|l|p{9cm}|}
		\hline
		\textbf{Team}                      & \textbf{Checks(Owner)}                                                                                                                                                 \\ \hline
		\textbf{Cics}                      & IBMCICS                                                                                                                                                                \\ \hline
		\textbf{Communication}             & IBMCS, CA\_TPX                                                                                                                                                         \\ \hline
		\textbf{Print}                     & CA\_DLVR, CA\_SPOOL, CA\_VIEW                                                                                                                                          \\ \hline
		\textbf{Rollout and Opperate(ROO)} & CA\_CSS, IBMCNZ,  IMBCTRACE, IBMDAE, IBMISPF, IBMXLOGR, IBMJES(2), IBMRRS, IBMRTM, IBMSDSF, IBMSDUMP, IBMSLIP, IBMSVA, IBMSYSTRACE, IBMTIMER, IBMTSOE,  IBMXCF, IBMGRS \\ \hline
		\textbf{Run Time Control(RTC)}     & IBMVLF, CA\_PMO, IBMRCF, IBMASM, IBMRSM, IBMSUP, IBMVSM,  CPWR\_THRUPUT\_MGR                                                                                           \\ \hline
		\textbf{Security}                  & CA\_ACF2, IBMICSF                                                                                                                                                      \\ \hline
		\textbf{SOE}                       & All checks starting with ZOSMIG                                                                                                                                        \\ \hline
		\textbf{Storage}                   & CA\_DISK, IBMALLOC, IBMCATALOG, IBMDMO, IBMHSM, IMBIOS, IBMOCE, IMBPDSE, IBMSMS, IBMVSAM(RLS), CA\_VANTAGE                                                             \\ \hline
		\textbf{zOpen}                     & IBMSSH, IBMUSS, IBMZFS                                                                                                                                                 \\ \hline
		\textbf{Databse(DB)}               & CA\_DB2, CA\_DTCM, CA\_IDMS                                                                                                                                            \\ \hline
		\textbf{Automation}                & IBMGDPS                                                                                                                                                                \\ \hline
	\end{tabular}
	\caption[Checks Per team]{Alle checkowners gegroepeerd per team binnen HCL}
	\label{tbl:Checks Per Team}
\end{table}

Nadat alle checks per team werden gegroepeerd, is er eerst zelf een voorstel gemaakt voor een de opstelling van hun checks waar zij feedback op konden geven. Wanneer er geen feedback was is de voorgestelde opstelling geimplementeerd.